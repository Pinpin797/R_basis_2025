%%%%
%
% attendance certificate for R training
%
%%%%
% use the corresponding paper size for your ticket definition
\documentclass[a4paper,12pt]{letter}

%% doc specifications and language
\usepackage[utf8]{inputenc}
\usepackage[T1]{fontenc}
\usepackage{textcomp}
\usepackage[french]{babel}
\usepackage[autolanguage]{numprint}
\usepackage[top=2cm, bottom=2cm, left=2cm, right=2cm,headheight=3cm,footskip=3cm,includeheadfoot]{geometry}
\usepackage{color}
\usepackage[hidelinks]{hyperref}
\usepackage{url}
\usepackage{svg}
\usepackage{multicol}

\setlength{\topskip}{0pt}

% load misc stuff
\usepackage{graphicx}

%%% manage data
\usepackage{datatool}

%%% first letter uppercase
\usepackage{titlecaps} % \titlecap{}
\newcommand{\firstcap}[1]{\titlecap{#1}}

%%% header-footer
\usepackage{fancyhdr}

%%%%%% info (to be filled)
\ifdefined\validator \else
  \def\validator{Mr Laurent Gilquin}
\fi
\ifdefined\validatorFunction \else
  \def\validatorFunction{ingénieur de recherche CNRS}
\fi
\ifdefined\organizer \else
  \def\organizer{le Conseil d'Analyse Numérique (CAN) de la SFR BioSciences (UAR3444/US8)}
\fi
\ifdefined\startDate \else
  \def\startDate{07/10/2024}
\fi
\ifdefined\endDate \else
  \def\endDate{31/01/2025}
\fi
\ifdefined\sessionDuration \else
  \def\sessionDuration{1h30}
\fi
\ifdefined\trainingTheme \else
  \def\trainingTheme{R débutant}
\fi
\ifdefined\trainingLocation \else
  \def\trainingLocation{ENS de Lyon (site Monod)}
\fi
\ifdefined\trainingSummary \else
  \def\trainingSummary{Cette formation est une introduction au langage de programmation R pour les personnes débutantes. Elle se déroule sur une douzaine de semaines à raison d’une heure et demi par semaine au semestre d'automne. Les participants et participantes apprendront les bases du langages R et comment utiliser l’IDE Rstudio. Ils apprendront ensuite à effectuer des manipulations de données complexes avec la suite de packages tidyverse, et enfin à produire des figures complexes à partir de données nettoyées.}
\fi
\ifdefined\trainingProgram \else
  \def\trainingProgram{\begin{multicols}{2}\begin{itemize}
\item Introduction à R et RStudio
\item Introduction à Tidyverse
\item Transformations avec ggplot2
\item Transformation de données
\item \textit{Pipping} et \textit{grouping}
\item Tidydata
\item \textit{String} \& \textit{RegExp}
\item \textit{Factors}
\end{itemize}\end{multicols}}
\fi
%%%%%%%%%%%%%%%%%%%%%%%%%%%%%%%%%%


%%%%%%%%%%%%%%%%%%%%%%%%%%%%%%%%

\begin{document}
\sffamily

\pagestyle{fancy}
\fancyhf{} % sets both header and footer to nothing
\renewcommand{\headrulewidth}{0pt}
\fancyhead[l]{\includegraphics[height=2.9cm]{logo/sfr.pdf}}
\fancyhead[r]{\sffamily SFR BioSciences\\ UCBL Gerland\\ 50 Av. Tony Garnier\\ 69366 LYON CEDEX 07, FRANCE\\ Tél : (+33) 4 37 28 74 70\\ \url{https://www.sfr-biosciences.fr/}}
\fancyfoot[c]{\makebox[\textwidth]{\includegraphics[height=2cm]{logo/ensl.pdf} \hfill \includegraphics[height=2cm]{logo/cnrs.pdf} \hfill \includegraphics[height=2cm]{logo/inserm.pdf} \hfill \includegraphics[height=2cm]{logo/ucbl.pdf}}}

\DTLloaddb{list}{participants.csv}
\DTLforeach{list}{%
\firstname=FIRSTNAME,\lastname=LASTNAME,\nsession=SESSION,\time=TIME}{%

\textbf{\Large Attestation individuelle de formation}\vspace{1em}

\begin{center}
\begin{minipage}{16cm}
Je, soussigné, \validator, \validatorFunction, atteste que \textbf{\firstname\ \lastname} a suivi la formation suivante, organisée par \organizer\ : \vspace{1em}

\begin{itemize}
\item Nom de la formation : \trainingTheme
\item Lieu de la formation : \trainingLocation
\item Dates de la formation : du \startDate\ au \endDate\ (\sessionDuration/semaine)
\item Durée totale de la formation : \time\ (soit \nsession\ séances)
\end{itemize} \vspace{1em}

\textbf{Objectifs de la formation~:} \vspace{1em}

\trainingSummary \vspace{1em}

\textbf{Programme de la formation~:}

\trainingProgram

\vspace{1cm}

\begin{flushright}
Fait à Lyon le \today

\vspace{1em}

\includegraphics[height=2cm]{signature}
\end{flushright}

\end{minipage}
\end{center}
}

\end{document}
